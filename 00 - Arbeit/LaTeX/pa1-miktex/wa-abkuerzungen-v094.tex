% 2012-03-22 Verwendung des optionalen Parameters f\"ur die Pluralform einer Abk\"urzung
%
% 2012-02-06 Umstellung auf die neuen Kommandos
%
%
%
%  J\"org Baumgart
%  Definition einiger Abk\"urzungen
%  


% Definition von Abk\"urzungen
%
% 1. Parameter: Schluessel (key) der Abkuerzung
% 2. Parameter: Abkuerzung
% 3. Parameter: Vollform
% 4. Parameter: Vollform im Plural (optional; falls nicht definiert, wird der Wert des dritten Parameters verwendet)
%
\seNewAcronymEntry{sdk}{SDK}{Software Development Kit}{Software Development Kits}

% 2012-03-24
% \"Uber den optionalen Parameter in eckigen Klammern wird die Pluralform f\"ur das erste 
% Auftreten der Abk\"urzung definiert

\seNewAcronymEntry[URLs]{url}{URL}{Uniform Resource Locator}%
{Uniform Resource Locators}






% Alternative Definition von Abk\"urzungen; diese sollten nicht verwendet werden!!!
%
%\newacronym{dhbw}{DHBW}{Duale Hochschule Baden-W\"urttemberg}
%\newacronym{usb}{USB}{Universal Serial Bus}


% Alternative Definition von Symbolen
%
% Achtung: ohne sort wird nach Name sortiert
%\newglossaryentry{pi}{
%name=$\pi$,
%description={Die Kreiszahl},
%type=symbolslist,
%sort=b
%}
%
%\newglossaryentry{ND}{
%name=$\mbox{\textsl{ND}}$,
%description={Nutzungsdauer einer Maschine},
%type=symbolslist,%
%sort=a
%}



% Alternative Definition von Glossareintr\"agen
%
%\newglossaryentry{glos:AD}{
%first=Active Directory\textsuperscript{GL},
%name=Active Directory,
%description={Active Directory ist in einem Windows 2000/Windows
%Server 2003-Netzwerk der Verzeichnisdienst, der die zentrale
%Organisation und Verwaltung aller Netzwerkressourcen erlaubt. Es
%erm\"oglicht den Benutzern \"uber eine einzige zentrale Anmeldung den
%Zugriff auf alle Ressourcen und den Administratoren die zentral
%organisierte Verwaltung, transparent von der Netzwerktopologie und
%den eingesetzten Netzwerkprotokollen. Das daf\"ur ben\"otigte
%Betriebssystem ist entweder Windows 2000 Server oder
%Windows Server 2003, welches auf dem zentralen
%Dom\"anencontroller installiert wird. Dieser h\"alt alle Daten des
%Active Directory vor, wie z.\,B. Benutzernamen und
%Kennw\"orter.\protect\seFootcite{Vgl.}{S. 200}{Dud09}}
%}














